\documentclass[10pt]{article}
\usepackage[final]{graphicx}
\usepackage{amsfonts}

\topmargin -.5in
\textwidth 6.6in
\textheight 9in
\oddsidemargin 0in
\usepackage{color}
\newcommand{\arvind}[1]{{\color{red}{Arvind: {#1}}}}

\def\ds{\displaystyle}
\def\d{\partial}
\linespread{1.5}

\begin{document}

\section{Summary and Future Work}
\subsection{Summary}

\subsection{Future Work}
Undoubtedly, there is a lot of work to be done further. Firstly, we would like to incorporate Hawaii sites to make the statistical model for predicting PM$_{2.5}$ using AOD measurements. Hawaii islands  are We would like make our model robust by training the data over several years.  We aim to make PM$_{2.5}$ predictions over the ocean using this model. With the predicted PM$_{2.5}$ values, we would make spatial plots. Spatial variations observed over a a time scale can provide insightful information about  intercontinental movement of PM$_{2.5}$. 

\begin{itemize}
\item Briefly summarize your contributions, and their possible
impact on the field (but don't just repeat the abstract or introduction).
\item Identify the limitations of your approach.
\item Suggest improvements for future work.
\item Outline open problems.
\end{itemize}

%~~~~~~~~~~~~~~REFERENCES~~~~~~~~~~~~~%
\begin{thebibliography}{99}

\bibitem{noaa} National Centers for Environmental Information. National Oceanic and Atmospheric Administration. Department of Commerce, n.d. Web. 23 July 2016. \textit{https://www.ncdc.noaa.gov/cdr/atmospheric/avhrr-aerosol-optical-thickness}.
\bibitem{epa} United States Environmental Protection Agency. AirData. EPA, 5 July 2016. Web. 23 July 2016. \textit{https://www3.epa.gov/airdata/}.

\bibitem{liu} Liu, Yang, Christopher J. Paciorek, and Petros Koutrakis. \textit{Estimating Regional Spatial and Temporal Variability of PM$_{2.5}$ Concentrations Using Satellite Data, Meteorology, and Land Use Information.} 6th ed. Vol. 117. N.p.: Environmental Health Perspectives, June 2009. Print.

\bibitem{lee} Lee, H J., Y Liu, B A. Coull, J Schwartz, and P Koutrakis. \textit{A novel calibration approach of MODIS AOD data to predict PM$_{2.5}$ concentrations.} N.p.: Atmospheric Chemistry and Physics, 2011. Print.

\bibitem{donk} Donkelaar, Aaron von, Randall V. Martin, Michael Brauer, and Brian L. Boys. \textit{Use of Satellite Observations for Long-Term Exposure Assessment of Global Concentrations of Fine Particulate Matter.} 2nd ed. Vol. 123. N.p.: Environmental Health Perspectives, 2015. Web. 26 July 2016. \textit{http://dx.doi.org/10.1289/ehp.1408646}

\bibitem{li} Li, Jing, Barbara E. Carlson, and Andrew A. Lacis. \textit{ How well do satellite AOD observations represent the spatial and temporal variability of PM$_{2.5}$ concentration for the United States?} N.p.: Atmospheric Environment, 2015. Print.

%http://disc.sci.gsfc.nasa.gov/giovanni/additional/users-manual/G3_manual_Chapter_19_AOT_comparison#sat_sensor

\end{thebibliography}

\end{document}
