\documentclass[10pt]{article}
\usepackage[final]{graphicx}
\usepackage{amsfonts}

\topmargin -.5in
\textwidth 6.6in
\textheight 9in
\oddsidemargin 0in

\def\ds{\displaystyle}
\def\d{\partial}
        
        \DeclareGraphicsExtensions{.pdf,.jpg,.png}
%\linespread{1.5}

\begin{document}




\section{The Approach}

One method we used to analyze trends in the PM$_{2.5}$ data was to make an animation to show the change in the concentration of PM$_{2.5}$  over time. The animation plots points at each site where the PM$_{2.5}$ data was collected, with color varying depending on the intensity of the reading, with darker colors indicating a larger concentration of PM$_{2.5}$. After analyzing this data we noticed (insert conclusion here... talk about it more...). The animation is available online (insert website-github?).


\end{document}
