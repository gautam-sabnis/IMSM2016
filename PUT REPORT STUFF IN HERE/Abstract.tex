\documentclass[10pt]{article}
\usepackage[final]{graphicx}
\usepackage{amsfonts}

\topmargin -.5in
\textwidth 6.6in
\textheight 9in
\oddsidemargin 0in
\usepackage{color}
\newcommand{\arvind}[1]{{\color{red}{Arvind: {#1}}}}

\def\ds{\displaystyle}
\def\d{\partial}
\linespread{1.5}

\begin{document}

\centerline{\large \bf Fusing surface and satellite-derived PM observations to determine the impact } 

\centerline{\large \bf of international transport on coastal PM $_{2.5}$ concentrations in the western U.S.}

\vspace{.1truein}

\def\thefootnote{\arabic{footnote}}
\begin{center}
  Neha Bora\footnote{Department of Mathematics, Iowa State University},
  Tuo Chen\footnote{Department of Statistics, University of Florida},
  Dana Cochran\footnote{Department of Mathematics, California State University Channel Islands},
  Kelly Dougan\footnote{Department of Mathematics, The State University of New York at Buffalo},
  Gautam Sabnis\footnote{Department of Statistics, Florida State University}
  Chuanping Yu \footnote{Industrial and System Engineering, Georgia Institute of Technology}
\end{center}

%\vspace{.1truein}

\begin{center}
Faculty Mentors: Mentor 1\footnote{Company},
Mentor 2\footnote{University}
\end{center}


\vspace{.3truein}
\centerline{\bf Abstract}

Long term exposure to PM$_{2.5}$ is associated with human health complications (insert citation). Surface readings of PM$_{2.5}$ in the states on the West Coast of the United States have reported to be higher than allowed by the Clean Air Act. One possible reason for this is international transport of air pollution on PM$_{2.5}$. This project explores the relationship between the surface readings of PM$_{2.5}$ from coastal sites with the AOD measurements from the AVHRR in these regions. Once we found a correlation between these readings, we implemented a model to approximate PM$_{2.5}$ concentrations in the Pacific ocean, where surface readings are not possible.


\end{document}